% Title:
% 	ARTICLE PLUS
% ----------------------
% Description:
% 	A template for scientific reports/articles.
%	Most necessary packages are imported,
%	so this should be a good starting point.
%
% Creator: Tommy O.

% -----------------------------------------------
% Package imports
% -----------------------------------------------
\documentclass[12pt, a4paper]{article} % 'twoside' for printing
\usepackage[utf8]{inputenc} % Allow input to be UTF-8
\usepackage[english]{babel} % Alternative: 'norsk' for norwegian
\usepackage{graphicx} % For importing graphics
\usepackage{amsthm, amsfonts, amssymb, amssymb} % All the AMS packages
\usepackage{mathtools} % Fixes a few AMS bugs
\usepackage{etoolbox} % Used to change mark at the end of thms
\usepackage[expansion=false]{microtype}% Fixes to make typography better
\usepackage[textsize=footnotesize]{todonotes} % Adds \todo{text} comments
\setlength{\marginparwidth}{2cm} % More space for the todo notes
\usepackage{hyperref} % For \href{URL}{text}
\usepackage{fancyhdr} % For fancy headers
\usepackage{cleveref} % Clever referencing using \cref
\usepackage[sharp]{easylist} % Easy nested lists using # (sharp)
\usepackage{parskip} % Web-like paragraphs, with spacing
\usepackage{multicol} % For multiple columns
\usepackage[linesnumbered,ruled]{algorithm2e} % For algorithms
\usepackage{tikz-cd} % For commutative diagrams
\usepackage{listings} % To include source-code
\usepackage{blindtext} % To generate lorem ipsum text
\usepackage[sc]{mathpazo} % A nice font, alternative to CM (default)
\usepackage[headings]{fullpage} % Make margins smaller
% \usepackage{geometry} % May be used to set margins, alternative to fullpage

% -----------------------------------------------
% Package setup
% -----------------------------------------------

% Add spacing to theorems and related environments
\newtheoremstyle{plainspaced}
{1em} % Space above
{1em} % Space below
{} % Body font
{} % Indent amount
{\bfseries} % Theorem head font
{.} % Punctuation after theorem head
{.5em} % Space after theorem head
{\thmname{#1}\thmnumber{ #2}\thmnote{ (#3)}} % Theorem head spec (can be left 
%empty, meaning `normal')

% Theorems, definition and examples
\theoremstyle{plainspaced} 
\newtheorem{theorem}{Theorem} 
\newtheorem{definition}{Definition}
\AtEndEnvironment{definition}{\null\hfill $\lrcorner$}
\newtheorem{example}{Example}
\AtEndEnvironment{example}{\null\hfill $\lrcorner$}

% Setup for the fancyhdr package
\rhead{\thepage}
\lhead{\nouppercase{\leftmark}}

% Section numbers in equations
\numberwithin{equation}{section} 

% -----------------------------------------------
% Misc settings
% -----------------------------------------------

% Spacing between easylist items
\newcommand{\listSpace}{-0.25em}

% Some math operators
\DeclareMathOperator{\R}{\mathbb{R}}
\DeclareMathOperator{\Z}{\mathbb{Z}}

% Make theorem and definition titles bold
\makeatletter
\def\th@plain{%
	\thm@notefont{}% same as heading font
	\itshape % body font
}
\def\th@definition{%
	\thm@notefont{}% same as heading font
	\normalfont % body font
}
\makeatother

% -----------------------------------------------
% Document variables
% -----------------------------------------------

\title{ARTICLE PLUS}
\author{Tommy O.}

% -----------------------------------------------
% Document start
% -----------------------------------------------

\begin{document}
\maketitle
\pagestyle{fancy}
\begin{abstract}
\todo{Write this.}
\blindmathtrue 
\blindtext[1]
\end{abstract}
\tableofcontents

% -----------------------------------------------
% Document content start
% -----------------------------------------------

\section{A first section}
\blindmathtrue 
\blindtext[1]
\begin{definition}[Group]
	\label{def:group}
	A group is a set $S$ along with an operation $\circ$
	with four axioms: identity, associativity, closure and inverse.
	\todo{Add citation.}
\end{definition}
There are two ways to cite the definition above.
Look at definition \ref{def:group}, or at \cref{def:group}.

\begin{theorem}[Pytagorean theorem]
	\label{thm:pyta}
	For a triangle with sides $a$, $b$ and $c$,
	it is true that $a^2 + b^2 = c^2$.
\end{theorem}
\begin{proof}
	Consider
	\begin{equation}
	\label{eqn:bayes_theorem}
		f_Y(y) = \int_{-\infty}^\infty f_Y(y\mid X=\xi )\,f_X(\xi)\,d\xi .
	\end{equation}
\end{proof}
Look closely at \cref{thm:pyta}, in other words 
\cref{eqn:bayes_theorem}.

\blindmathtrue 
\blindtext[1]


\section{Lists}
Here is a list:
\begin{easylist}[itemize]
	\ListProperties(Space=\listSpace, Space*=\listSpace)
	# But if the rebellion is to be successful.
	# not to  between two, it is necessary.
	# think that there is no.
	# When a rational conviction has.
\end{easylist}

Another list:
\begin{easylist}[enumerate]
	\ListProperties(Space=\listSpace, Space*=\listSpace)
	# think that there is no.
	# When a rational conviction has.
	# in oneself whatever.
\end{easylist}

\begin{equation}
	T = \frac{1}{2}m v^2
\end{equation}

\section{Diagrams}
Consider the following diagram, which shows the relationship between grad, curl 
and div
\begin{equation*}
\begin{tikzcd}
f \arrow{r}{\text{grad}} \arrow[swap, bend right=30]{rr}{0} & \vec{F} 
\arrow[bend left=40]{rr}{0}
\arrow{r}{\text{curl}}  &  \vec{F} 
\arrow{r}{\text{div}}  & \mathbb{R}  .
\end{tikzcd}
\end{equation*}


\section{The bibliography}

A .bib file will contain the bibliographic information of our document.

A .bib file will contain the bibliographic information of our document. I will only give a simple example, since there are many tools to generate the entries automatically.
\todo{Write more about this.}

A .bib file will contain the bibliographic information of our document. I will only give a simple example, since there are many tools to generate the entries automatically. I will not explain the structure of the file itself. Oh well.

% -----------------------------------------------
% ---- BIBLIOGRAPHY
% -----------------------------------------------

\bibliographystyle{apalike} % 'alpha' is also good
%\bibliography{bibliography} % Reference to 'bibliography.bib'
\end{document}


