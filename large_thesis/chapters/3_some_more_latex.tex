\label{chapter:some_more_latex}

\section{References and lists}

Although \cite{strang_linear_1976} is a great introduction to linear algebra, \cite{roman_advanced_2005} presents the material in a more abstract way.

Here's a list of some linear algebra concepts.
\begin{easylist}[enumerate]
	\ListProperties(Space=\listSpace, Space*=\listSpace, Numbers1=a, Numbers2=r)
	
	# Linear algebra is about matrices and vectors.
	
	## The elements are usually numbers from $\R$ or $\C$.
	## Matrices are given by two indices, vectors by one.
	
	# Matrix multiplication is given by $y_i = \sum_j A_{ij} x_j$.
	
	# If $A = a_{ij}$, then the tranpose flips across the diagonal so that $A^T = a_{ji}$.
	The transpose is the adjoint, i.e. $\innerprod{Ax}{y} = \innerprod{x}{A^Ty}$.
\end{easylist}


Here's a different type of list.

\begin{easylist}[enumerate]
	\ListProperties(Space=\listSpace, Space*=\listSpace, Numbers=r, FinalMark={)})
	
	# Linear algebra is about matrices and vectors.
	
	## The elements are usually numbers from $\R$ or $\C$.
	## Matrices are given by two indices, vectors by one.

\end{easylist}

Here's a different type of list.

\begin{easylist}[itemize]
	\ListProperties(Space=\listSpace, Space*=\listSpace)
	
	# Linear algebra is about matrices and vectors.
	
	## The elements are usually numbers from $\R$ or $\C$.
	## Matrices are given by two indices, vectors by one.
	
	# Matrix multiplication is given by $y_i = \sum_j A_{ij} x_j$.
	
	# If $A = a_{ij}$, then the tranpose flips across the diagonal so that $A^T = a_{ji}$.
	The transpose is the adjoint, i.e. $\innerprod{Ax}{y} = \innerprod{x}{A^Ty}$.
\end{easylist}


\section{Algorithms}

Here's an algorithm.

\begin{algorithm}
	\label{algo:student_algorithm}
	\SetKwInOut{Input}{Input}
	\SetKwInOut{Output}{Output}
	%\DontPrintSemicolon % \PrintSemicolon

	\Input{Student $s$.}
	\Output{Inverse $s^{-1}$ such that $s \circ s^{-1} = s^{-1} \circ s = \text{Id}$.}
	\BlankLine
	
	\tcp{Morning.}
	Grab coffee\;
	\While{thesis not finished}{
		Read mathematics\;
		Write thesis\;
	}
	Relax for an hour\;
	\BlankLine
	\tcp{Evening.}
	\For{every friend $f \in F$}{
		Call $f$\tcp*{Give your friends a call.} \label{algline:call_friends}
		}
	Sleep\;
	\caption{Algorithm for a thesis.}
\end{algorithm}

Notice in Algorithm \ref{algo:student_algorithm} above that the student does in fact call his friends in line \ref{algline:call_friends}.

Here's some Python code.

\lstinputlisting[language = iPython, firstline=9]{./code/example_code/composition.py}


\section{title}
\blindtext

\subsection{title}
\blindtext

\subsubsection{title}
\blindtext

\subsubsection{title}
\blindtext

\section{title}
\blindtext
\subsection{title}
\blindtext

\subsubsection{title}
\blindtext
